\documentclass[12pt,a4paper]{report}

% Essential packages
\usepackage[utf8]{inputenc}
\usepackage[T1]{fontenc}
\usepackage{times}
\usepackage[margin=1in]{geometry}
\usepackage{graphicx}
\usepackage{amsmath,amssymb}
\usepackage{booktabs}
\usepackage{multirow}
\usepackage{caption}
\usepackage{hyperref}
\usepackage{float}
\usepackage{enumitem}
\usepackage{xcolor}
\usepackage{setspace}
\usepackage{fancyhdr}

% Hyperref setup
\hypersetup{colorlinks=true,linkcolor=blue,citecolor=blue}

% Header/Footer
\pagestyle{fancy}
\fancyhf{}
\rhead{BTP Report - IIT Kharagpur}
\lhead{Commodity Price Prediction}
\cfoot{\thepage}

% Line spacing
\onehalfspacing

\begin{document}

% ============================================================================
% TITLE PAGE
% ============================================================================
\begin{titlepage}
    \centering
    \vspace*{0.5cm}
    
    {\LARGE\bfseries\color{blue!80!black} COMMODITY PRICE PREDICTION SYSTEM}\\[0.5cm]
    
    {\large\bfseries Using Machine Learning and Deep Learning}\\[0.5cm]
    
    {\large\itshape Bachelor's Thesis Project Report (BTP1)}\\[1.2cm]
    
    Submitted by:\\
    \textbf{Gaurav Kumar(22AG36012)}\\[0.8cm]
    
    Under the supervision of:\\
    \textbf{Dr. Priyabrata Pradhan}\\[0.8cm]
    
    \textbf{Department of Agriculture and Food Engineering}\\
    \textbf{IIT Kharagpur}\\[1cm]
    
    {\Large\bfseries INDIAN INSTITUTE OF TECHNOLOGY}\\[0.3cm]
    {\Large\bfseries Kharagpur}\\[0.8cm]
    
    \includegraphics[width=4cm]{"download (2).png"}\\[1cm]
    
    {\bfseries Agricultural and Food Engineering Department}\\
    {\bfseries Indian Institute of Technology Kharagpur}\\[0.5cm]
    
    {\color{blue}\bfseries Autumn Semester, 2025-26}\\[0.3cm]
    {\color{blue}\bfseries November 27, 2025}
    
\end{titlepage}

% ============================================================================
% CERTIFICATE & ABSTRACT
% ============================================================================
\chapter*{Certificate}
\addcontentsline{toc}{chapter}{Certificate}

This is to certify that the project titled \textbf{``Commodity Price Prediction Using Machine Learning and Deep Learning''} submitted by \textbf{Gaurav Kumar (22AG36012)} to IIT Kharagpur is a record of bonafide work carried out under my supervision.

\vspace{2cm}
\begin{flushright}
    \textbf{Dr. Priyabrata Pradhan}\\
    Department of Agriculture and Food Engineering, IIT Kharagpur\\
    Date: November 27, 2025
\end{flushright}

\chapter*{Abstract}
\addcontentsline{toc}{chapter}{Abstract}

This project develops a machine learning system for predicting agricultural commodity prices in West Bengal, India. Using historical data from 2014-2025 comprising 173,094 records across 18 districts and 61 markets, we implement two models: \textbf{XGBoost} and \textbf{Deep Neural Network}.

The XGBoost model achieves \textbf{5.43\% MAPE} with \textbf{R\textsuperscript{2} = 0.9453}, while the Neural Network achieves \textbf{4.64\% MAPE} with \textbf{R\textsuperscript{2} = 0.9601}. For 2025 predictions, the Neural Network achieves excellent \textbf{4.44\% MAPE} with \textbf{R\textsuperscript{2} = 0.9724}. The system predicts prices for Rice, Jute, and Wheat with 7-day forecasting capability.

A web application built with Flask and React provides an accessible interface for stakeholders to obtain price predictions.

\textbf{Keywords:} Machine Learning, XGBoost, Neural Networks, Price Prediction, Agriculture

\tableofcontents

% ============================================================================
% CHAPTER 1: INTRODUCTION
% ============================================================================
\chapter{Introduction}

\section{Background}

Agriculture employs over 50\% of India's workforce. Price volatility significantly impacts farmers' livelihoods, leading to distress selling and financial instability. West Bengal, a major agricultural state, produces significant quantities of rice, jute, and wheat with complex pricing patterns.

\section{Problem Statement}

Develop an accurate commodity price prediction system that:
\begin{enumerate}[noitemsep]
    \item Predicts prices for Rice, Jute, and Wheat in West Bengal
    \item Provides 7-day ahead forecasts
    \item Incorporates weather, economic indicators, and historical patterns
    \item Offers an accessible web interface
\end{enumerate}

\section{Objectives}

\begin{enumerate}[noitemsep]
    \item Compile comprehensive historical price data (2014-2025)
    \item Design relevant features capturing temporal and economic patterns
    \item Implement XGBoost and Neural Network models
    \item Develop a production-ready web application
\end{enumerate}

% ============================================================================
% CHAPTER 2: METHODOLOGY
% ============================================================================
\chapter{Methodology}

\section{Dataset}

\begin{table}[H]
\centering
\caption{Dataset Overview}
\begin{tabular}{@{}ll@{}}
\toprule
\textbf{Parameter} & \textbf{Value} \\
\midrule
Total Records & 177,320 \\
Time Period & 2014-2025 (11 years) \\
Districts & 18 \\
Markets & 61 \\
Commodities & Rice, Jute, Wheat \\
Database Size & 51.14 MB \\
\bottomrule
\end{tabular}
\end{table}

\begin{table}[H]
\centering
\caption{Commodity Distribution}
\begin{tabular}{@{}lrr@{}}
\toprule
\textbf{Commodity} & \textbf{Records} & \textbf{\%} \\
\midrule
Rice & 130,572 & 75.4\% \\
Jute & 34,425 & 19.9\% \\
Wheat & 8,097 & 4.7\% \\
\bottomrule
\end{tabular}
\end{table}

Data sources include Agmarknet (prices), IMD (weather), RBI (economic indicators), and Ministry of Agriculture (MSP, production data).

\section{Feature Engineering}

We designed 36 features in the following categories:

\textbf{Temporal Features:} Year, month, day, quarter, day of week, weekend indicator, seasonal flags (monsoon, winter, summer).

\textbf{Categorical Features:} District, market, commodity, and variety (label encoded).

\textbf{Economic Indicators:} CPI, per capita income, food subsidy, MSP (Minimum Support Price).

\textbf{Agricultural Parameters:} Temperature, rainfall, area, production, yield, fertilizer consumption, export/import data.

\textbf{Derived Features:} Temperature-rainfall interaction, production efficiency ratios, CPI-MSP ratio.

\textbf{Price Statistics:} Commodity average, market average, district-commodity average, monthly averages.

\section{Models}

\subsection{XGBoost (Extreme Gradient Boosting)}

XGBoost builds an ensemble of decision trees sequentially, optimizing:
\begin{equation}
\mathcal{L}(\phi) = \sum_{i=1}^{n} l(y_i, \hat{y}_i) + \sum_{k=1}^{K} \Omega(f_k)
\end{equation}
where $l$ is the loss function and $\Omega$ is the regularization term.

\textbf{Key Hyperparameters:} 1000 estimators, max depth 8, learning rate 0.05, L1/L2 regularization, GPU acceleration.

\subsection{Neural Network}

A 5-layer deep neural network with architecture:
\begin{itemize}[noitemsep]
    \item Input: 36 features
    \item Hidden layers: 256 $\rightarrow$ 128 $\rightarrow$ 64 $\rightarrow$ 32 $\rightarrow$ 16 neurons
    \item Activation: ReLU with BatchNorm and Dropout
    \item Output: 1 neuron (price prediction)
    \item Optimizer: Adam (learning rate 0.001)
    \item Early stopping with patience 20
\end{itemize}

\section{Evaluation Metrics}

\begin{itemize}[noitemsep]
    \item \textbf{MAE:} Mean Absolute Error (Rs)
    \item \textbf{RMSE:} Root Mean Squared Error (Rs)
    \item \textbf{MAPE:} Mean Absolute Percentage Error (\%)
    \item \textbf{R\textsuperscript{2}:} Coefficient of Determination
\end{itemize}

% ============================================================================
% CHAPTER 3: RESULTS
% ============================================================================
\chapter{Results and Analysis}

\section{Model Performance}

\begin{table}[H]
\centering
\caption{Model Performance Comparison}
\begin{tabular}{@{}lrrrr@{}}
\toprule
\textbf{Metric} & \textbf{XGBoost} & \textbf{Neural Network} \\
\midrule
MAE (Rs) & 167.42 & \textbf{143.73} \\
RMSE (Rs) & 285.36 & \textbf{253.66} \\
MAPE (\%) & 5.43 & \textbf{4.64} \\
R\textsuperscript{2} Score & 0.9453 & \textbf{0.9601} \\
\bottomrule
\end{tabular}
\end{table}

\begin{table}[H]
\centering
\caption{Prediction Accuracy Distribution}
\begin{tabular}{@{}lrr@{}}
\toprule
\textbf{Error Threshold} & \textbf{XGBoost} & \textbf{Neural Network} \\
\midrule
Within 5\% & 89.2\% & \textbf{92.4\%} \\
Within 10\% & 97.5\% & \textbf{98.1\%} \\
Within 15\% & 99.2\% & \textbf{99.5\%} \\
\bottomrule
\end{tabular}
\end{table}

\section{Commodity-wise Performance}

\begin{table}[H]
\centering
\caption{MAPE by Commodity (\%)}
\begin{tabular}{@{}lrr@{}}
\toprule
\textbf{Commodity} & \textbf{XGBoost} & \textbf{Neural Network} \\
\midrule
Rice & 5.21 & \textbf{4.42} \\
Jute & 5.89 & \textbf{4.95} \\
Wheat & 5.45 & \textbf{4.68} \\
\bottomrule
\end{tabular}
\end{table}

\section{Validation Results}

Sample predictions compared against actual database values:

\begin{table}[H]
\centering
\caption{Sample Predictions vs Actual Prices}
\begin{tabular}{@{}llrrr@{}}
\toprule
\textbf{Commodity} & \textbf{District} & \textbf{Actual} & \textbf{XGBoost} & \textbf{NN} \\
\midrule
Wheat & North 24 Parganas & Rs 2,060 & Rs 2,180 & Rs 2,105 \\
Jute & Murshidabad & Rs 5,880 & Rs 6,120 & Rs 5,894 \\
Rice & Medinipur(W) & Rs 3,800 & Rs 3,950 & Rs 3,811 \\
\bottomrule
\end{tabular}
\end{table}

\section{Feature Importance}

Top features influencing predictions (XGBoost):
\begin{enumerate}[noitemsep]
    \item Commodity average price (18.7\%)
    \item Market average price (15.6\%)
    \item MSP - Minimum Support Price (13.4\%)
    \item Variety average price (9.8\%)
    \item CPI (8.7\%)
\end{enumerate}

\section{Discussion}

\textbf{Neural Network Advantages:}
\begin{itemize}[noitemsep]
    \item Superior accuracy (4.64\% MAPE vs 5.43\%)
    \item Excellent 2025 predictions (4.44\% MAPE, R\textsuperscript{2} = 0.9724)
    \item Effectively captures temporal and seasonal patterns
    \item Better generalization on unseen data
\end{itemize}

\textbf{XGBoost Observations:}
\begin{itemize}[noitemsep]
    \item Good performance (5.43\% MAPE)
    \item Handles categorical features effectively
    \item Provides feature importance insights
    \item Useful as fallback model
\end{itemize}

% ============================================================================
% CHAPTER 4: SYSTEM IMPLEMENTATION
% ============================================================================
\chapter{System Implementation}

\section{Architecture}

The system follows a three-tier architecture:
\begin{itemize}[noitemsep]
    \item \textbf{Presentation Layer:} React.js frontend
    \item \textbf{Application Layer:} Flask REST API
    \item \textbf{Data Layer:} SQLite database + trained models
\end{itemize}

\section{Technology Stack}

\begin{table}[H]
\centering
\caption{Technology Stack}
\begin{tabular}{@{}ll@{}}
\toprule
\textbf{Component} & \textbf{Technology} \\
\midrule
Frontend & React.js 18 \\
Backend & Flask 3.1, Waitress WSGI \\
ML Framework & XGBoost 3.1.2, TensorFlow 2.20 \\
Database & SQLite \\
Language & Python 3.10 \\
\bottomrule
\end{tabular}
\end{table}

\section{Web Application Features}

\begin{enumerate}[noitemsep]
    \item \textbf{Model Selection:} Choose between XGBoost or Neural Network
    \item \textbf{Input Form:} Select commodity, district, market, variety, date
    \item \textbf{7-Day Forecast:} Display predictions for current day + 6 days
    \item \textbf{Responsive Design:} Works on desktop and mobile devices
\end{enumerate}

\section{API Endpoints}

\begin{table}[H]
\centering
\caption{REST API}
\begin{tabular}{@{}lll@{}}
\toprule
\textbf{Endpoint} & \textbf{Method} & \textbf{Description} \\
\midrule
/ & GET & Main page \\
/predict & POST & Get price predictions \\
/get\_markets & POST & Markets for district \\
/get\_varieties & POST & Varieties for commodity \\
\bottomrule
\end{tabular}
\end{table}

% ============================================================================
% CHAPTER 5: CONCLUSION
% ============================================================================
\chapter{Conclusion}

\section{Summary}

This project successfully developed a commodity price prediction system achieving:
\begin{itemize}[noitemsep]
    \item Neural Network with \textbf{4.64\% MAPE} and \textbf{R\textsuperscript{2} = 0.9601}
    \item XGBoost model with \textbf{5.43\% MAPE} and \textbf{R\textsuperscript{2} = 0.9453}
    \item 2025 predictions: Neural Network achieves \textbf{4.44\% MAPE}, \textbf{R\textsuperscript{2} = 0.9724}
    \item Coverage of 18 districts, 61 markets, 3 commodities
    \item Production-ready web application
\end{itemize}

\section{Contributions}

\begin{enumerate}[noitemsep]
    \item Novel feature set combining economic and agricultural indicators
    \item Comparative analysis of gradient boosting vs deep learning
    \item Deployed system accessible to farmers and policymakers
\end{enumerate}

\section{Limitations}

\begin{itemize}[noitemsep]
    \item Limited to West Bengal region
    \item Does not account for sudden policy changes or disasters
    \item Requires continuous data updates
\end{itemize}

\section{Future Work}

\begin{enumerate}[noitemsep]
    \item Implement LSTM for better temporal pattern capture
    \item Extend coverage to pan-India
    \item Add more commodities (vegetables, pulses)
    \item Develop mobile application
    \item Integrate real-time weather data APIs
\end{enumerate}

% ============================================================================
% REFERENCES
% ============================================================================
\begin{thebibliography}{9}

\bibitem{xgboost}
Chen, T., \& Guestrin, C. (2016). XGBoost: A Scalable Tree Boosting System. \textit{KDD}.

\bibitem{lstm}
Hochreiter, S., \& Schmidhuber, J. (1997). Long Short-Term Memory. \textit{Neural Computation}.

\bibitem{dropout}
Srivastava, N. et al. (2014). Dropout: Preventing Overfitting. \textit{JMLR}.

\bibitem{adam}
Kingma, D. P., \& Ba, J. (2014). Adam Optimizer. \textit{arXiv:1412.6980}.

\bibitem{rf}
Breiman, L. (2001). Random Forests. \textit{Machine Learning}.

\bibitem{tensorflow}
Abadi, M. et al. (2016). TensorFlow. \textit{OSDI}.

\bibitem{sklearn}
Pedregosa, F. et al. (2011). Scikit-learn. \textit{JMLR}.

\end{thebibliography}

\end{document}
